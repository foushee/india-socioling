\documentclass[9pt,twocolumn,twoside,lineno]{article}
%\documentclass{foushee-adapted-preprint}
\usepackage{multirow}
\usepackage{paralist}
\usepackage{booktabs}
\usepackage{microtype}
\usepackage{rotating}
\usepackage{caption}
\sidecaptionvpos{figure}{t}
\leadauthor{Foushee} 
%\setlength{\marginparwidth}{2cm}
%{\renewcommand{\baselinestretch}{0.5}\selectfont#2\par}
\newcommand{\fulllanguagestab}{S1}

\newcommand{\geomeanstab}{S2}
\newcommand{\religionmeanstab}{S3}
\newcommand{\wealthmeanstab}{S4}
\newcommand{\faceaudiomeanstab}{S5}
\newcommand{\facelabelmeanstab}{S6}
\newcommand{\learningmeanstab}{S7}


\newcommand{\geomodeltab}{S8}
\newcommand{\religionmodeltab}{S9}
\newcommand{\religionchirelmodeltab}{S10}
\newcommand{\faceaudiomodeltab}{S11}
\newcommand{\facelabelmodeltab}{S12}

\newcommand{\facemodalitytab}{S13}
\newcommand{\faceenglishmodalitytab}{S14}

\newcommand{\wealthmodeltab}{S15}
\newcommand{\learningmodeltabs}{S16--S20}

\newcommand{\geofamtab}{S21}
\newcommand{\relfamtab}{S22}
\newcommand{\faceaudiofamtab}{S23}
\newcommand{\facelabelfamtab}{S24}
%\newcommand{\wealthfamtab}{S25}
%\newcommand{\learningfamtab}{S21}

\newcommand{\geogendertab}{S25}
\newcommand{\relgendertab}{S26}
\newcommand{\wealthgendertab}{S27}
\newcommand{\faceaudiogendertab}{S28}
\newcommand{\facelabelgendertab}{S29}
%\newcommand{\wealthfamtab}{S25}

%FIGURES
\newcommand{\nonengbarplot}{S1}
\newcommand{\engbarplot}{S2}
\newcommand{\geomeanstab}{\ref{tab:geographic-origin-means}}
\newcommand{\religionmeanstab}{\ref{tab:religion-means}}
\newcommand{\wealthmeanstab}{\ref{tab:wealth-means}}
\newcommand{\faceaudiomeanstab}{\ref{tab:face-audio-means}}
\newcommand{\facelabelmeanstab}{\ref{tab:face-label-means}}
\newcommand{\learningmeanstab}{\ref{tab:face-learning-means}}

\begin{document}
\title{Sociolinguistic development in a diverse, multilinguistic environment: Evidence from multilingual children in Gujarat, India}
\shorttitle{{Socio-Multilinguistic Development}}

\author[1,2, \Letter]{Ruthe Foushee}
\author[2]{Sophie Regan}
\author[2]{Roya Baharloo}
\author[2]{Mahesh Srinivasan}

\affil[1]{Department of Psychology, New School for Social Research, 80 $5^{\textnormal{th}}$ Avenue, New York, NY 10011}
\affil[2]{Department of Psychology, University of California, Berkeley, 2121 Berkeley Way West, Berkeley, CA 95720}
\maketitle
%\addbibresource{bibliography.bib}
%0.9\baselineskip
\begin{abstract}
\noindent
In today's pluralistic societies, children regularly acquire multiple languages and are exposed to an even larger set of languages spoken by others in their environment. Yet despite the prevalence of multilingualism globally, most research on sociolinguistic development has focused on monolingual children in environments with relatively little linguistic diversity, and as such has left questions of what children take different languages to socially signify largely unaddressed. The present study aimed to fill this gap by tracing the development of social inferences about different languages among 129 multilingual 7- to 13-year-olds in Gujarat, India. Contrary to the prediction that children in multilingual contexts should be unlikely to make stereotyping inferences about a person speaking a language (e.g., because they might expect the person to know additional languages), children in our sample selectively linked the different languages and language varieties that we probed (Gujarati, Marathi, Hindi, Urdu, Tamil, American English, Indian English, and Mandarin Chinese) with different social dimensions---including facial appearance, geographic origin, religion, and wealth. Children's responses generally reflected associations grounded in real-world regularities, but also reflected some associations that do not have a real-world basis (e.g., judging that Indian English speakers tend to be white, Christian, and originate from outside of India). Older children were also more likely to predict different languages to be differentially learnable by individuals of specific ethnicities, exhibiting a kind of essentialist belief. We discuss our findings in light of the sociolinguistic study of \textit{personae}.
\end{abstract}
\begin{corrauthor}
ruthe\at newschool.edu
\end{corrauthor}
\begin{keywords}
\noindent
\textit{multilingualism} | \textit{sociolinguistic development} | \textit{linguistic essentialism} | \textit{ linguistic diversity} | \textit{personae}
\end{keywords}
\maketitle
\vspace{5pt}
%\section*{Introduction}
\lettrine{I}n today's increasingly pluralistic societies, children regularly acquire multiple languages and are routinely exposed to an even larger set of languages spoken by others in their community \parencite{lin2020research, grosjean2010bilingual}. 
For example, children in Gujarat, India---the site of the present study---often acquire or have significant exposure to the local regional language (e.g., Gujarati), official languages of the government and schools (e.g., Hindi and Indian English), languages spoken by family members who may have originated from other regions in India (e.g., Marathi, Punjabi, Tamil), and languages spoken by religious sub-communities (e.g., Urdu), among others. Given the widespread presence of linguistic diversity in children's environments, it is important to understand how children make sense of such variation: What kinds of social inferences do children make about individuals on the basis of the language(s) they speak, and how might this change over development? %as they grow older
\end{document}
